\documentclass[a4paper, 10pt]{article}

\usepackage[hidelinks, pageanchor=false]{hyperref}
\usepackage[top=0.5cm, bottom=0.5cm, left=1cm, right=1cm]{geometry}
\usepackage{enumitem}
\usepackage{adjustbox}

\begin{document}

\input style.tex
\input personalInfos.tex

%Beggining of CV

\begin {identity}
    \name {\personalFirstName}{\personalLastName}
    \contact {\personalAddress \linebreak
    \personalPhoneNumber, \href{mailto:\personalMail}{\personalMail}\linebreak
    Nationality: French\linebreak
    Github: \url{https://github.com/grassead}\linebreak
    LinkedIn: \url{\personalLinkedin}}
\end {identity}

\title {Embedded software engineer - Android / Graphics / Virtualisation}

\begin{section} {Work experience}
    \begin{experience}{February 2017}{Present}{Technical Expert}{Smile Open Source Solutions}{Asni\`{e}res-sur-Seine}{France}
    \begin{subexperience}
    \begin{itemize}[parsep=0cm,itemsep=0cm,topsep=0cm]
		\item \textbf {Software architecture:}
		\begin{itemize}[parsep=0cm,itemsep=0cm,topsep=0cm]
		\item Graphics and Virtualization architect in a project involving two virtual machines with display sharing.
		\end{itemize}
	\end{itemize}
	\end{subexperience}
	\end{experience}

    \begin{experience}{May 2015}{January 2017}{Senior embedded virtualization engineer}{RedBend}{Saint-Quentin-en-Yvelines}{France}
	\begin{subexperience}
	    \begin{itemize}[parsep=0cm,itemsep=0cm,topsep=0cm]
		\item \textbf {Graphics team:}
		\begin{itemize}[parsep=0cm,itemsep=0cm,topsep=0cm]
		    \item Debugging in both user and kernel space;
		    \item Design of the new virtualized powerHAL for Android 5.1;
		    \item Studying how to handle GPU and display hardware of an Intel processor into two different virtual machines (One under QNX 6.5, the other one under Linux (AGL)):
		    \begin{itemize}[parsep=0cm,itemsep=0cm,topsep=0cm]
			\item Creation of a Linux distribution from scratch using Build Root to start the project;
			\item Integration of Intel Linux distribution using Yocto;
			\item Studying Linux kernel drivers and documentation to understand the Intel Gen 9 Graphics technology;
			\item Modification of Intel's Linux Kernel drivers;
		    \end{itemize}
		    \item Participating on porting the graphics virtualization solution from Android to QNX 7.0:
		    \begin{itemize}[parsep=0cm,itemsep=0cm,topsep=0cm]
			\item Studying the feasibility on QNX 6.6;
			\item Design and implementation of the new Graphics Buffer Allocator;
			\item Integration of the whole development.
		    \end{itemize}
		\end{itemize}
	    \end{itemize}
	\end{subexperience}
    \end{experience}

    \begin{experience}{January 2011}{April 2015}{Embedded software engineer Android/Linux}{Parrot}{Paris}{France}
	\begin{itemize}[parsep=0cm,itemsep=0cm,topsep=0cm]
	    \begin{subexperience}
		\item \textbf {Android:}
		    \begin{itemize}[parsep=0cm,itemsep=0cm,topsep=0cm]
			\item Porting versions Gingerbread (2.3.7), JellyBean (4.2 \& 4.4), Lollipop (5.0) on Parrot7 SoC (C / C++):
			    \begin{itemize}[parsep=0cm,itemsep=0cm,topsep=0cm]
				\item Writing Hardware Abstraction Layer for different subsystems:
				    \begin{itemize}[parsep=0cm,itemsep=0cm,topsep=0cm]
					\item Video out (V4L2 userspace) (hwcomposer);
					\item Video encoding / decoding (OMX);
					\item Memory allocator (gralloc).
				    \end{itemize}
				\item Proposal of Parrot source code integration in Android source tree;
				\item Patch submission to Google;
				\item Working with providers to integrate their software (ARM / On2 / Cinemo)
			    \end{itemize}
			\item Framework modification and extension on Gingerbread (2.3.7) and JellyBean (4.2) (Java / C / C++):
			    \begin{itemize}[parsep=0cm,itemsep=0cm,topsep=0cm]
				\item Integration of Bluetooth Stack from Parrot to Android 2.3.7 at framework Level;
				\item Design an implementation of a Radio Layer Interface to integrate messaging and telephony features to Android 2.3;
				\item Integration of Oculus Rift (DK1 and DK2) to Android 4.2.
			    \end{itemize}
			\item Debugging on Cupcake (1.5), Gingerbread (2.3.7), Jellybean (4.2) and Lollipop (5.0) (Java / C / C++).
		    \end{itemize}
	    \end{subexperience}
	    \begin{subexperience}
		\item \textbf {Linux Kernel:}
		    \begin{itemize}[parsep=0cm,itemsep=0cm,topsep=0cm]
			\item Development:
			    \begin{itemize}[parsep=0cm,itemsep=0cm,topsep=0cm]
				\item Implementation of a driver to process video stream from RAM to RAM;
				\item Back porting some drivers from Android Linux Driver to Parrot version;
				    \begin{itemize}[parsep=0cm,itemsep=0cm,topsep=0cm]
					\item Contiguous Memory Allocator;
					\item Android ION;
					\item Android synchronisation mechanism (Fences);
					\item Modification from Samsung for V4L2 framework;
					\item Integration of ARM Mali (GPU) driver and userspace
				    \end{itemize}
				\item Upgrade kernel version from 3.4 to 3.10 for Android Lollipop. This include many modifications in Parrot drivers;
				\item Implementation of Board Support Package for some products.
			    \end{itemize}
			\item Debug:
			    \begin{itemize}[parsep=0cm,itemsep=0cm,topsep=0cm]
				\item Parrot V4L2 drivers;
				\item HDMI out driver, support to pass the EMC certification;
				\item ARM Mali400 driver;
				\item Contiguous Memory Allocator;
				\item Android ION.
			    \end{itemize}
		    \end{itemize}
	    \end{subexperience}
	\end{itemize}
    \end{experience}

    \begin{unbreakableExperience}{March 2010}{September 2010}{Internship}{Trusted Logic}{Versailles}{France}
	\begin{subexperience}
	    \begin{itemize}[parsep=0cm,itemsep=0cm,topsep=0cm]
		\item Tools and Tests for a Web server embedded in a smart card. (Jython / Java / J2Me / JavaCard)
		    \begin{itemize}[parsep=0cm,itemsep=0cm,topsep=0cm]
			\item Writing tools in order to handle a web server located on a smart card.
			    \begin{itemize}[parsep=0cm,itemsep=0cm,topsep=0cm]
				\item Design of a BIP \textless==\textgreater TCP/IP gateway in Java/J2Me ;
				\item Implementation of a JSP like engine that generate JavaCard code;
				\item Implementation of test applets.
			    \end{itemize}
		    \end{itemize}
	    \end{itemize}
	\end{subexperience}
    \end{unbreakableExperience}

    \begin{unbreakableExperience}{June 2009}{September 2009}{Internship}{Research lab of ESEO}{Angers}{France}
	\begin{subexperience}
	    \begin{itemize}[parsep=0cm,itemsep=0cm,topsep=0cm]
		\item Study of NFC protocol. Design of a demonstrator containing a phone and a NFC reader.
	    \end{itemize}
	\end{subexperience}
    \end{unbreakableExperience}
\end{section}

\begin{unbreakableSection} {Education}
    \begin{education}{2005}{2010}{Engineering degree in embedded software}{ESEO}{Angers}{France}
    \end{education}
\end{unbreakableSection}

\begin{unbreakableSection} {Personal projects}
    \begin{projects}
	\begin{itemize}[parsep=0cm,itemsep=0cm,topsep=0cm]
	    \item \href{https://github.com/grassead/snesemu}{Design of a Super Nintendo emulator in C}
	    \item \href{https://github.com/grassead/sudokusolver}{Design of a Sudoku Solver in C++}
	    \item \href{https://github.com/grassead/memorychecker}{Design of a Memory Allocation checker in C}
	    \item \href{https://github.com/grassead/cv}{Up-to-date resume}
	\end{itemize}
    \end{projects}
\end{unbreakableSection}

\begin{section} {Skills}
    \begin{skills}
	\begin{itemize}[parsep=0cm,itemsep=0cm,topsep=0cm]
	    \item Android Porting
	    \item Android Framework
	    \item Android Application
	    \item ARM Mali
	    \item Build Root
	    \item C/C++
	    \item GIT
	    \item Intel Gen 9 Graphics
	    \item Java, J2Me, JavaCard
	    \item Latex
	    \item Linux Kernel Driver Development
	    \item Linux Kernel Contiguous memory management
	    \item OMX
	    \item OpenGL ES
	    \item QNX
	    \item V4L2
	    \item Virtualisation
	    \item Wayland
	    \item Yocto
	\end{itemize}
    \end{skills}
\end{section}

\begin{section} {Langages}
    \begin{languages}
	\begin{itemize}[parsep=0cm,itemsep=0cm,topsep=0cm]
	    \item French: Native speaker
	    \item English: FCE, TOEIC 865, Intermediate level
	\end{itemize}
    \end{languages}
\end{section}

\end{document}

